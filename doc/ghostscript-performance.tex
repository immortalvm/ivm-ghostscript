\documentclass[a4paper,11pt]{article}
\usepackage[utf8]{inputenc}
\usepackage[T1]{fontenc}
\usepackage[includeheadfoot,vmargin=2cm,hmargin=2cm]{geometry}
\usepackage{times}
\usepackage{booktabs}
\usepackage[small]{titlesec}
\usepackage{enumitem}
\usepackage{amsmath}
\usepackage{amsthm}
\usepackage{mathptmx}
\usepackage{tikz}
\usepackage{xifthen}
\usepackage{fancyvrb}
\frenchspacing
\setlist{nosep}

\renewcommand{\ttdefault}{cmtt}
\newcommand{\code}[1]{\texttt{#1}}
\newenvironment{codeblock}{\begin{trivlist}\ttfamily
  \item }{\end{trivlist}}
\DefineVerbatimEnvironment{Diff}{Verbatim}{formatcom=\footnotesize}

\author{Thor Kristoffersen}
\date{\today}
\title{Ghostscript PDF Rendering Performance\\on the iVM Virtual Machine}
\begin{document}

\maketitle

\section{Introduction}

This is a report on the performance of the ivm64 port of Ghostscript for high-resolution rendering of PDF documents.

PDF rendering is the prime use case for the iVM virtual machine in the VirtuMa project.
The project objective is to be able to render one page in less than three minutes on current hardware.

\section{Test Setup}
\label{sec:test setup}

This section describes what was tested and how the testing was performed.

\subsection{Test Documents}

Three documents of different, typical characteristics were tested: a technical document, a government report, and a book.

In each test case, Adobe Acrobat Pro DC version 2020.012.20048 was used to convert the original document to PDF/A format.

\subsection{Test Platform}

The tests were performed on a VirtualBox virtual machine running on an Intel Core i7-8850H (2.60GHz).

The Ghostscript source code was compiled with ivm64-gcc version 1.0rc5-ar based on gcc 8.3.0, then assembled with the iVM assembler version 0.3, and then executed on an optimized C implementation of the iVM.

\subsection{Test Execution}

The rendering of each PDF document was done by executing the Ghostscript executable with the following command-line, which renders the document at 300 ppi, with 16 levels of antialiasing for both text and graphics.
\begin{trivlist}\ttfamily
\item gs -sDEVICE=ivm64 -dGraphicsAlphaBits=4 -dTextAlphaBits=4 -r300x300 \\ -dNOPAUSE -dBATCH -dSAFER -sOutputFile=- Input.pdf
\end{trivlist}
Since the output from the virtual machine was rendered as a sequence of PNG files, it was possible to infer the rendering time for each page, allowing per-page statistical parameters to be calculated.

\section{Test Results}

This section reports the results of the tests.

\subsection{A 17-Page Technical Document}

This test case is the project deliverable \emph{How to Build a Machine}, a 17-page technical document containing tables and mathematical formulas, but no graphics.
The test results are shown in Figure~\ref{tab:17-page-doc}.

\begin{table}[htbp]
  \centering\small
  \begin{tabular}{@{}lr@{}}
    \hline
    Parameter      & Time [s] \\
    \hline
    Page mean      & 32  \\
    Page median    & 30  \\
    Page minimum   & 6   \\
    Page maximum   & 36  \\
    Page std.dev.  & 8   \\
    Document total & 537 \\
    \hline
  \end{tabular}
  \caption{Test results for a 17-page mathematical document}
  \label{tab:17-page-doc}
\end{table}

\subsection{A 96-Page Government Report with Diagrams}

This test case is \emph{NOU 2019:11 Enklere merverdiavgift med én sats} by the Norwegian government, a 96-page report, of which 16 pages contain diagrams and other graphics elements.
The test results are shown in Figure~\ref{tab:96-page-doc}.

\begin{table}[htbp]
  \centering\small
  \begin{tabular}{@{}lr@{}}
    \hline
    Parameter      & Time [s] \\
    \hline
    Page mean      & 86   \\
    Page median    & 83   \\
    Page minimum   & 6    \\
    Page maximum   & 604  \\
    Page std.dev.  & 60   \\
    Document total & 8299 \\
    \hline
  \end{tabular}
  \caption{Test results for a 96-page government report}
  \label{tab:96-page-doc}
\end{table}

\subsection{A 378-page Book with Illustrations}

This test case is \emph{A Tale of Two Cities} by Charles Dickens, a 378-page e-book, of which 32 pages contain illustrations.
The test results are shown in Figure~\ref{tab:378-page-doc}.

\begin{table}[htbp]
  \centering\small
  \begin{tabular}{@{}lr@{}}
    \hline
    Parameter      & Time [s] \\
    \hline
    Page mean      & 58    \\
    Page median    & 63    \\
    Page minimum   & 7     \\
    Page maximum   & 99    \\
    Page std.dev.  & 18    \\
    Document total & 21780 \\
    \hline
  \end{tabular}
  \caption{Test results for a 378-page book}
  \label{tab:378-page-doc}
\end{table}

\section{Conclusion}

On current hardware the mean PDF rendering time per page for typical document types at high resolution is well below the project requirement of three minutes.

\end{document}
